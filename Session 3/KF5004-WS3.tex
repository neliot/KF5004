\documentclass[11pt]{article}
%\usepackage[framemethod=tikz]{mdframed}
\usepackage{tcolorbox}
\usepackage{titling}
\usepackage[os=win]{menukeys}
\newcommand{\numpy}{{\tt numpy}}    % tt font for numpy

\topmargin -.5in
\textheight 9in
\oddsidemargin -.25in
\evensidemargin -.25in
\textwidth 7in

\pretitle{%
  \begin{center}
  \LARGE
  \includegraphics[width=15cm]{../images/logo.png}\\[\bigskipamount]
}
\posttitle{\end{center}}

\begin{document}

% ========== Edit your name here
\author{Dr. Neil Eliot / Dr. Alun Moon}
\title{KF5004\\------\\Workshop 3\\------\\Setting up a basic \texttt{DNS} Caching Server\\------}
\date{September 2019}
\maketitle

\newpage

% \begin{center}
%     \noindent\rule{8cm}{0.4pt}
% \end{center}

% ========== Introduction

\noindent\textbf{Learning Outcomes:}
\begin{itemize}
    \item Create a Headless server.
    \item Install \texttt{BIND9}.
    \item Configure \texttt{BIND9} as a Caching Server with \texttt{Forwarders}.
    \item Identify the network activity generated by a \texttt{DNS} lookup.
\end{itemize}

% \begin{center}
% \noindent\rule{8cm}{0.4pt}
% \end{center}

\begin{tcolorbox}[title={\textbf{Important:}}]
    Before you begin, ensure the lab. \texttt{PC} you are using is connected to the off campus lab infrastructure and the machine receives an \texttt{IP} address from the lab. \texttt{DHCP} Server.
\end{tcolorbox}
\newpage

\noindent\textbf{The Exercise}\\
\begin{tcolorbox}[colback=blue!20]
    \noindent\textbf{From the commands and processes defined in the lecture carry out the following:}
\end{tcolorbox}

% ========== Begin answering questions here

\begin{tcolorbox}[title={\textbf{NOTE:}}]
    In this workshop you should use your virtual \texttt{Ubuntu Desktop} that you create in the first workshop. \textbf{Please do not alter the base machines in the laboratory}.
\end{tcolorbox}

\begin{enumerate}
    \item Create a new Virtual Machine and carry out a basic install. (Choose which ever technique you prefer)
        \begin{itemize}
            \item Create a machine from scratch.
            \item Copy the basic server create in the last session.
            \item Use the pre-created image from the \texttt{NAS} drive.
        \end{itemize}
    \item Install \texttt{BIND9} on the server via a remote connection. 
        \begin{tcolorbox}[title={\textbf{NOTE:}}]
            Make sure you install both \texttt{BIND9} and the \texttt{DNS} utilities. 
        \end{tcolorbox}
    \item Configure the machine to use two possible servers as \texttt{Forwarders}.
        \begin{tcolorbox}[title={\textbf{NOTE:}}]
            Please use the addresses \texttt{192.168.101.29} and \texttt{192.168.101.30}.
        \end{tcolorbox}
    \item Test some queries against your \texttt{DNS} server and others.
        \begin{tcolorbox}[title={\textbf{NOTE:}}]
            \begin{itemize}
                \item In \texttt{Wireshark} you can expand \texttt{DNS} packets to identify the queries and responses.
                \item To capture the traffic effectively you need to run \texttt{Wireshark} on the base Windows machine and monitor all ethernet traffic.
                \item Wnen querying your own Virtual \texttt{DNS} server it should be running on a different machine.
            \end{itemize}
        \end{tcolorbox}
        \begin{tcolorbox}[colback=blue!20]
            In this section you need to:
            \begin{itemize}
                \item Use \texttt{Wireshark} to monitor the network.
                \item Identify which server is being communicated with and the \texttt{DNS} traffic that is generated. 
            \end{itemize}
        \end{tcolorbox}
        \begin{itemize}
            \item Set your \texttt{Virtual client} to use the nameserver \texttt{8.8.8.8}.
                \begin{itemize}
                    \item \texttt{\$nslookup www.google.com}
                    \item \texttt{\$nslookup www.northumbria.ac.uk}
                    \item \texttt{\$nslookup www.offcampusnetwork.co.uk}
                    \item \texttt{\$nslookup www.vbox.org.uk}
                \end{itemize} 
            \item Set your \texttt{Virtual client} to use the nameservers \texttt{192.168.101.29, 192.168.101.30}.
                \begin{itemize}
                    \item \texttt{\$nslookup www.google.com}
                    \item \texttt{\$nslookup www.northumbria.ac.uk}
                    \item \texttt{\$nslookup www.offcampusnetwork.co.uk}
                    \item \texttt{\$nslookup www.vbox.org.uk}
                    \item \texttt{\$nslookup www.vbox.org.uk 8.8.8.8}
                \end{itemize} 
            \item Set your \texttt{Virtual client} to use the nameserver you created.
                \begin{itemize}
                    \item \texttt{\$nslookup www.google.com}
                    \item \texttt{\$nslookup www.northumbria.ac.uk}
                    \item \texttt{\$nslookup www.offcampusnetwork.co.uk}
                    \item \texttt{\$nslookup www.vbox.org.uk}
                    \item \texttt{\$nslookup www.vbox.org.uk 8.8.8.8}
                \end{itemize} 
        \end{itemize}
    \end{enumerate}
END    
\end{document}