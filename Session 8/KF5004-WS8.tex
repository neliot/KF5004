\documentclass[11pt]{article}
%\usepackage[framemethod=tikz]{mdframed}
\usepackage{tcolorbox}
%\usepackage{figure}
\usepackage[os=win]{menukeys}
\newcommand{\numpy}{{\tt numpy}}    % tt font for numpy

\topmargin -.5in
\textheight 9in
\oddsidemargin -.25in
\evensidemargin -.25in
\textwidth 7in

\begin{document}

% ========== Edit your name here
\author{Dr. Neil Eliot / Dr. Alun Moon}
\title{KF5004\\------\\Workshop 8\\------\\Setting up \texttt{Apache} Server\\------}
\date{September 2019}
\maketitle

\newpage

% \begin{center}
%     \noindent\rule{8cm}{0.4pt}
% \end{center}

% ========== Introduction

\noindent\textbf{Learning Outcomes:}
\begin{itemize}
    \item Configure \texttt{Apache}:
        \begin{itemize}
            \item Per User.
            \item Load-balanced.
        \end{itemize}
    \item Configure \texttt{DNS} Round Robin.
\end{itemize}

% \begin{center}
% \noindent\rule{8cm}{0.4pt}
% \end{center}

\begin{tcolorbox}[title={\textbf{Important:}}]
    Before you begin, ensure the lab. \texttt{PC} you are using is connected to the off campus lab infrastructure and the machine receives an \texttt{IP} address from the lab. \texttt{DHCP} Server.
\end{tcolorbox}
\newpage

\noindent\textbf{The Exercise}\\
\begin{tcolorbox}[colback=blue!20]
    \noindent\textbf{From the commands and process defined in the notes from the lecture carry out the following:}
\end{tcolorbox}

% ========== Begin answering questions here

\begin{tcolorbox}[title={\textbf{NOTE:}}]
    In this workshop you will need your \texttt{DNS} architecture from last practical.
\end{tcolorbox}

\begin{enumerate}
    \item Create a new \texttt{Virtual Server} 
        \begin{tcolorbox}[title={\textbf{NOTE:}}]
            \noindent Don't forget servers \textbf{SHOULD} be statically addressed.
        \end{tcolorbox}
    \item Install a full \texttt{LAMP} installation on the new server.
    \item Set up mappings for your new server (Forward and Reverse).
        \begin{tcolorbox}[title={\textbf{NOTE:}}]
            \noindent The network address you are using is \texttt{192.168.0.0/16}
        \end{tcolorbox}
        \begin{itemize}
            \item \texttt{<your server IP>} $\Leftrightarrow$ \texttt{www1.student.co.uk}
            \item \texttt{<your server IP>} $\Leftrightarrow$ \texttt{www.student.co.uk}
        \end{itemize}
    \item Access the default web page.
    \item Create a static "HELLO" page that includes the \texttt{IP} Address of the server and a \texttt{PHP} test page (\texttt{info.php}).
        \begin{itemize}
            \item Access the two pages on your new server using the 2 \texttt{URL}s
        \end{itemize}
    \item Configure your server to support Per-User websites and create a simple "HELLO" page (\texttt{index.html}) in the student account.
        \begin{itemize}
            \item The page must be located in the \texttt{public\_html} directory.
        \end{itemize}
    \item Create another user account (\texttt{ceo}) and enable \texttt{PHP} scripting for only that account.
        \begin{itemize}
            \item Copy the PHP test page into both the Per-User account directories and access them from a browser.
                \begin{tcolorbox}[title={\textbf{NOTE:}}]
                    \noindent The \texttt{ceo} account should return the \texttt{info} page and the student account should list the \texttt{PHP} code.
                \end{tcolorbox}
            \end{itemize}
    \item Create a further 2, statically addressed, \texttt{LAMP} Servers.
        \begin{itemize}
            \item Create a "HELLO" page on both the servers that includes their \texttt{IP} Address.
        \end{itemize}
    \item Update your \texttt{www.student.co.uk} \texttt{URL} to map the 3 \texttt{IP} Addresses of the \texttt{LAMP} servers using \texttt{Round-Robin} (cyclic).
        \begin{itemize}
            \item Access the "HELLO" page from 3 different clients.
        \end{itemize}
\end{enumerate}    
END
\end{document}