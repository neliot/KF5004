\documentclass[11pt]{article}
%\usepackage[framemethod=tikz]{mdframed}
\usepackage{tcolorbox}
\usepackage[os=win]{menukeys}
\newcommand{\numpy}{{\tt numpy}}    % tt font for numpy

\topmargin -.5in
\textheight 9in
\oddsidemargin -.25in
\evensidemargin -.25in
\textwidth 7in

\begin{document}

% ========== Edit your name here
\author{Dr. Neil Eliot / Dr. Alun Moon}
\title{KF5004\\------\\Workshop 5\\------\\Setting up a \texttt{Reverse Zone} on a \texttt{Primary DNS Server}\\------}
\date{September 2019}
\maketitle

\newpage

% \begin{center}
%     \noindent\rule{8cm}{0.4pt}
% \end{center}

% ========== Introduction

\noindent\textbf{Learning Outcomes:}
\begin{itemize}
    \item Create a \texttt{Reverse zone}.
    \item Identify the network activity generated by a \texttt{rDNS} lookup.
\end{itemize}

% \begin{center}
% \noindent\rule{8cm}{0.4pt}
% \end{center}

\begin{tcolorbox}[title={\textbf{Important:}}]
    Before you begin, ensure the lab. \texttt{PC} you are using is connected to the off campus lab infrastructure and the machine receives an \texttt{IP} address from the lab. \texttt{DHCP} Server.
\end{tcolorbox}
\newpage

\noindent\textbf{The Exercise}\\
\begin{tcolorbox}[colback=blue!20]
    \noindent\textbf{From the commands and process defined in the notes from the lecture carry out the following:}
\end{tcolorbox}

% ========== Begin answering questions here

\begin{tcolorbox}[title={\textbf{NOTE:}}]
    In this workshop you should use your virtual \texttt{Ubuntu Desktop} that you create in the first workshop. \textbf{Please do not alter the base machines in the laboratory}.
\end{tcolorbox}

\begin{enumerate}
    \item Create a new \texttt{Virtual Machine} and carry out a basic install. (Choose which ever technique you prefer)
        \begin{itemize}
            \item Create a machine from scratch.
            \item Copy the basic server create in the last session.
            \item Use the pre-created image from the \texttt{NAS} drive.
        \end{itemize}
        \begin{tcolorbox}[title={\textbf{NOTE:}}]
            A \texttt{DNS} Server should always have a static \texttt{IP} Address so once the machine has booted ensure you use one of your reserved \texttt{IP} Addresses.
        \end{tcolorbox}
    \item Install \texttt{BIND9} on the server via a remote connection. 
        \begin{tcolorbox}[title={\textbf{NOTE:}}]
            Make sure you install both \texttt{BIND9} and the \texttt{DNS} utilities. 
        \end{tcolorbox}
    \item Configure the server to support a \texttt{rDNS} zone for the addresses in last weeks \texttt{student.co.uk zone} (follow the lecture notes but use your \texttt{IP} range).
        \begin{itemize}
            \item Last weeks mappings listed below:
            \begin{itemize}
                \item \texttt{192.168.100.2} $\Rightarrow$ \texttt{www2.student.co.uk}
                \item \texttt{192.168.100.2} $\Rightarrow$ \texttt{www.student.co.uk}
                \item \texttt{<your client IP>} $\Rightarrow$ \texttt{me.student.co.uk}
                \item \texttt{192.168.100.21} $\Rightarrow$ \texttt{server.student.co.uk}
                \item \texttt{192.168.100.2} $\Rightarrow$ \texttt{mail.student.co.uk}
                \item \texttt{192.168.100.254} $\Rightarrow$ \texttt{gateway.student.co.uk}
            \end{itemize}
        \end{itemize}
    \item Test your \texttt{zone} file has no errors.
    \item Test the following queries:
        \begin{tcolorbox}[colback=blue!20]
            \noindent\textbf{For all of these exercises make sure you run \texttt{Wireshark} on the host \texttt{PC} and capture ALL Ethernet Traffic. You should be able to identify when caching is working from a lack of network traffic. An note the response addresses}
        \end{tcolorbox}
        \begin{itemize}
            \item \texttt{192.168.100.2}
            \item \texttt{192.168.100.1}
            \item \texttt{8.8.8.8}
        \end{itemize}
    \item \texttt{DNS} Load-balancing and \texttt{rDNS}:
    \begin{tcolorbox}[title={\textbf{NOTE:}}]
        \noindent Don't forget. When making changes to a \texttt{zone} you must update the \texttt{serial number} and restart the service.
    \end{tcolorbox}
    \begin{itemize}
            \item Update your \texttt{student.co.uk zone} and map 3 \texttt{IP} addresses (listed below) to \texttt{www}. (addresses listed below)
                \begin{itemize}
                    \item 192.168.101.2
                    \item 192.168.101.3
                    \item 192.168.101.4
                \end{itemize}
            \item Create \texttt{rDNS} entries for the load-balanced \texttt{www IP} addresses.
            \item Query the \texttt{rDNS} for the \texttt{3 IP} addresses you just added.
        \end{itemize}
    \item Limit the access to your \texttt{rDNS} domain so only locally connected clients can access it and run a query from your \texttt{Virtual Client}. 
    \item Block the \texttt{IP} address of your \texttt{Virtual Client} and try running the query again.
        \begin{itemize}
            \item Check the response from the server in \texttt{Wireshark}.
        \end{itemize}
\end{enumerate}    

\end{document}