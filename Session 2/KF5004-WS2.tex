\documentclass[11pt]{article}
%\usepackage[framemethod=tikz]{mdframed}
\usepackage{tcolorbox}
\usepackage[os=win]{menukeys}
\newcommand{\numpy}{{\tt numpy}}    % tt font for numpy

\topmargin -.5in
\textheight 9in
\oddsidemargin -.25in
\evensidemargin -.25in
\textwidth 7in

\begin{document}

% ========== Edit your name here
\author{Dr. Neil Eliot / Dr. Alun Moon}
\title{KF5004\\------\\Workshop 2\\------\\Setting up a basic headless \texttt{Ubuntu Server}\\------}
\date{September 2019}
\maketitle

\medskip

\begin{center}
    \noindent\rule{8cm}{0.4pt}
\end{center}

% ========== Introduction

\noindent\textbf{Learning Outcomes:}
\begin{itemize}
    \item Understand what a Headless server is.
    \item Configure a server to support \texttt{ssh} and \texttt{telnetd}.
    \item Convert a server from dynamic to a static \texttt{IP} Addressing.
    \item Understand the location and purpose of the networking configuration files of a \texttt{18.04} server.
\end{itemize}

\begin{center}
\noindent\rule{8cm}{0.4pt}
\end{center}

\begin{tcolorbox}[title={\textbf{Note:}}]
    Before you begin ensure you have the lab. \texttt{PC} you are using connected to the lab infrastructure and the machine receives an \texttt{IP} address from the lab. \texttt{DHCP} Server.
\end{tcolorbox}

\noindent\textbf{The Exercise}\\

\noindent \textbf{From the commands and process shown in the lecture notes, carry out the following:\\}

% ========== Begin answering questions here
\begin{enumerate}
    \item Create a new Virtual Machine and carry out a basic install.
        \begin{itemize}
            \item Take the default settings in most cases.
                \begin{itemize}
                    \item Set your machine name to your \texttt{User ID}.
                    \item Set Username to \texttt{student}
                    \item Set Password to \texttt{northumbria}
                    \item Set memory to \texttt{2Gb}
                    \item Set Network card to Bridged.
                \end{itemize}
        \end{itemize}
    \item Identify the address the server has been given by the lab. \texttt{DHCP} Server (you will change this later). 
        \begin{tcolorbox}[title={\textbf{Notes:}}]
            \begin{itemize}
                \item At this stage you should close the virtual machine down and copy the folder it is contained in. 
                \item This will be your base machine for future work!
                \item To `shutdown' your server use the command: \texttt{sudo shutdown -h now} 
            \end{itemize}
        \end{tcolorbox}
    \item Restart your machine to continue.
    \item Modify the server so that it has a static configurations of: 
        \begin{itemize}
            \item \texttt{IP - <choose one from your batch>}
            \item \texttt{Subnet - 255.255.0.0 (/16)}
            \item \texttt{Gateway - 192.168.100.254}
            \item \texttt{DNS - 192.168.101.29}
                \begin{tcolorbox}[title={\textbf{Notes:}}]
                    \begin{itemize}
                        \item At this stage you are using the \texttt{Virtual} lab. name servers.
                        \item Later you will be creating your own \texttt{DNS} architecture.
                    \end{itemize}
                \end{tcolorbox}
        \end{itemize}
    \item Add \texttt{ssh} services to your machine for remote access (\texttt{openssh-server}).
        \begin{tcolorbox}[title={\textbf{Note:}}]
            In this next section you should connect to your virtual server from a different machine on the network. This forces the traffic outside of the host machines \texttt{TCP} stack and onto the physical network so \texttt{Wireshark} can capture the traffic.
        \end{tcolorbox}
    \end{enumerate}
    \noindent \textbf{Using the server you have created carry out the following:}
    \begin{enumerate}
        \setcounter{enumi}{5}
        \item Test remote connectivity to your virtual \texttt{Ubuntu server}.
            \begin{tcolorbox}[title={\textbf{Notes:}}]
                \begin{itemize}
                    \item \texttt{putty} should be on all the lab \texttt{PC}'s Windows installation.
                    \item If you are using \texttt{Ubuntu desktop} you should be able to connect to the server by using the \texttt{ssh} command from a \texttt{Linux} console.\\
                    \item On the first connection you will need to accept the \texttt{ssh} certificate.
                \end{itemize}
            \end{tcolorbox}
        \item Using \texttt{Wireshark} monitor the network and identify the communications to/from the virtual machine and identify the \texttt{IP} address and \texttt{MAC} address in the \texttt{IP} headers.
        \item Install \texttt{telnet} server (\texttt{telnetd}).
        \item Run \texttt{wireshark} and capture the traffic while you are logging in using either the \texttt{telnet} command in \texttt{Linux} or \texttt{telnet} protocol in \texttt{putty} (filter \texttt{tcp.port == 23}).
        \item Examine the traffic and identify the \texttt{TCP} transactions for the login.
            \begin{itemize}
                \item You will find that each letter of the username is visible as it is transmitted from the client to the server and echoed back from the server.
                \item The password is not echoed you will see the characters only once. From the client to the server.
            \end{itemize}
        \item Shutdown the server.
            \begin{itemize}
                \item \texttt{\$shutdown -h now}
            \end{itemize}
    \end{enumerate}
    \begin{tcolorbox}[title={\textbf{TIME PERMITTING:}}]
        Research how to change the student password and how to add another user.
    \end{tcolorbox}
\end{document}