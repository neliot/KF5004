\documentclass[11pt]{article}
%\usepackage[framemethod=tikz]{mdframed}
\usepackage{tcolorbox}
\usepackage{titling}
%\usepackage{figure}
\usepackage[os=win]{menukeys}
\newcommand{\numpy}{{\tt numpy}}    % tt font for numpy

\topmargin -.5in
\textheight 9in
\oddsidemargin -.25in
\evensidemargin -.25in
\textwidth 7in

\pretitle{%
  \begin{center}
  \LARGE
  \includegraphics[width=15cm]{../images/logo.png}\\[\bigskipamount]
}
\posttitle{\end{center}}

\begin{document}

% ========== Edit your name here
\author{Dr. Neil Eliot / Dr. Alun Moon}
\title{KF5004\\------\\Workshop 10\\------\\Setting up \texttt{MySQL}\\------}
\date{September 2019}
\maketitle

\newpage

% \begin{center}
%     \noindent\rule{8cm}{0.4pt}
% \end{center}

% ========== Introduction

\noindent\textbf{Learning Outcomes:}
\begin{itemize}
    \item Configure \texttt{MySQL} server.
    \item Configure \texttt{MySQL} client.
    \item Configure \texttt{PHPMyAdmin}.
    \item Create users and databases for Websites.
\end{itemize}

% \begin{center}
% \noindent\rule{8cm}{0.4pt}
% \end{center}

\begin{tcolorbox}[title={\textbf{Important:}}]
    Before you begin, ensure the lab. \texttt{PC} you are using is connected to the off campus lab infrastructure and the machine receives an \texttt{IP} address from the lab. \texttt{DHCP} Server.
\end{tcolorbox}
\newpage

\noindent\textbf{The Exercise}\\
\begin{tcolorbox}[colback=blue!20]
    \noindent\textbf{From the commands and process defined in the notes from the lecture carry out the following:}
\end{tcolorbox}

% ========== Begin answering questions here

\begin{tcolorbox}[title={\textbf{NOTE:}}]
    In this workshop you will need your \texttt{DNS} and Web server architecture from the last practical.
\end{tcolorbox}

\begin{enumerate}
    \item Using the \texttt{DNS} server by adding/updating the following records.
        \begin{itemize}
            \item Add the following records forward lookups to your \texttt{DNS} Architecture (\texttt{A} records).
                \begin{itemize}
                    \item \texttt{<IP OF THE MYSQL SERVER>} $\Rightarrow$ \texttt{mysql.student.co.uk}
                    \item \texttt{<IP YOUR FIRST WEBSERVER>} $\Rightarrow$ \texttt{www1.student.co.uk}
                    \item \texttt{<IP YOUR SECOND WEBSERVER>} $\Rightarrow$ \texttt{www2.student.co.uk}
                    \item \texttt{<IP YOUR THIRD WEBSERVER>} $\Rightarrow$ \texttt{www3.student.co.uk}
                    \item \texttt{<IP YOUR FIRST WEBSERVER>} $\Rightarrow$ \texttt{www.student.co.uk}
                    \item \texttt{<IP YOUR SECOND WEBSERVER>} $\Rightarrow$ \texttt{www.student.co.uk}
                    \item \texttt{<IP YOUR THIRD WEBSERVER>} $\Rightarrow$ \texttt{www.student.co.uk}
                \end{itemize}
            \item Add the following reverse records to your \texttt{DNS} Architecture (\texttt{PTR} records).
                \begin{itemize}
                    \item \texttt{www1.student.co.uk}
                    \item \texttt{www2.student.co.uk}
                    \item \texttt{www3.student.co.uk}
                    \item \texttt{www.student.co.uk}
                \end{itemize}
        \end{itemize}
    \item Create a new \texttt{Virtual Server} with a static \texttt{IP} address.
        \begin{itemize}
            \item Install \texttt{MySQL server} and test you can connect to it locally.
            \item Install \texttt{PHPMyAdmin} and test you can connect to it from your \texttt{Virtual Client}.
            \item Create a \texttt{user} and \texttt{database} of the same name \texttt{TEST} with the user password set to \texttt{Northumbria2109!}
        \end{itemize}
    \item Carry out the following on all three web servers (They should already be using \texttt{NFS} from last week):
        \begin{itemize}
            \item Install the \texttt{MySQL client} software.
            \item Create a \texttt{MySQL} test script on each server.
        \end{itemize}
    \item Test the connectivity of each web server by going to their individual \texttt{URL/test script}.
    \item Test the load balanced access to the servers: info pages, shared area, and \texttt{MySQL} test script from three different clients.
        \begin{itemize}
            \item Use \texttt{Wireshark} and analyse the connectivity of each server to identify which clients and servers are connecting to each other.
        \end{itemize}
\end{enumerate}    
END
\end{document}