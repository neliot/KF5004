\documentclass[11pt]{article}
\usepackage{tcolorbox}
\usepackage{titling}
\usepackage[os=win]{menukeys}
\newcommand{\numpy}{{\tt numpy}}    % tt font for numpy

\topmargin -.5in
\textheight 9in
\oddsidemargin -.25in
\evensidemargin -.25in
\textwidth 7in

\pretitle{%
  \begin{center}
  \LARGE
  \includegraphics[width=15cm]{../images/logo.png}\\[\bigskipamount]
}
\posttitle{\end{center}}

\begin{document}

% ========== Edit your name here
\author{Dr. Neil Eliot / Dr. Alun Moon}
\title{KF5004\\------\\Workshop 1\\------\\Setting up a Machine in \texttt{VMWare}\\------}
\date{September 2019}
\maketitle

\newpage

% \begin{center}
%     \noindent\rule{8cm}{0.4pt}
% \end{center}

% ========== Introduction

\paragraph{Learning Outcomes:}
\begin{itemize}
\item Understand what a \texttt{Virtual Machine} is.
\item Understand how to create a \texttt{Virtual Machine}.
\item Run \texttt{Wireshark} on the local network adapter.
\item Analyse traffic from the local network and a \texttt{Virtual PC}
\end{itemize}

% \begin{center}
% \noindent\rule{8cm}{0.4pt}
% \end{center}

\begin{tcolorbox}[title={\textbf{Important:}}]
    Before you begin, ensure the lab. \texttt{PC} you are using is connected to the off campus lab infrastructure and the machine receives an \texttt{IP} address from the lab. \texttt{DHCP} Server.
\end{tcolorbox}
\newpage

\noindent\textbf{The Exercise}\\
\begin{tcolorbox}[colback=blue!20]
    \noindent\textbf{From the commands and process defined in the notes from the lecture carry out the following:}
\end{tcolorbox}

\begin{enumerate}
    \item Ensure your portable storage device is \texttt{NTFS} format.
    \begin{itemize}
        \item Why?
    \end{itemize} 
    \item Create a new \texttt{Virtual Machine} and carry out a basic \texttt{Ubuntu Desktop} install.
        \begin{itemize}
            \item Take the default settings.
            \item Set the Network Card to \texttt{Bridged}.
            \begin{tcolorbox}[title={\textbf{Note:}}]
                If there are issues check the \texttt{Virtual Network} settings are using the correct Ethernet card.
            \end{tcolorbox}
            \item Set your machine name to your \texttt{User ID}.    
            \item Set the first account to be \texttt{student} with a password of \texttt{northumbria}.    
        \end{itemize}
    \item Identify the address the client \texttt{PC} has been given by the \texttt{DHCP} Server.
    \begin{tcolorbox}[colback=blue!20]
        \textbf{At this stage you can `shutdown' the \texttt{Virtual Machine} and copy it. This machine will now be the client machine you use for the rest of the module (although you will have to change the \texttt{IP} settings). A basic \texttt{Ubuntu Client} virtual machine will be made available from the \texttt{NAS} drive in the event you loose or corrupt your own.}
    \end{tcolorbox}
    \item Start the copy of your machine and check the \texttt{IP} settings.
    \item Using \texttt{Wireshark} monitor the network and identify the communications between your \texttt{Virtual Machine} and the local network gateway (\texttt{192.168.100.254}).
        \begin{itemize}
            \item A simple ping test will generate enough traffic. 
            \begin{tcolorbox}[colback=blue!20]
                \textbf{(Use \keys{CTRL+ALT+t} to open a terminal in your \texttt{Virtual Client})}
            \end{tcolorbox}
        \end{itemize}
    \item Alter the \texttt{MAC} Address of the \texttt{Virtual Machine} and check the network again.
\end{enumerate}
END
\end{document}
