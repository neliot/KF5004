\documentclass[11pt]{article}
\usepackage{tcolorbox}
\usepackage[os=win]{menukeys}
\newcommand{\numpy}{{\tt numpy}}    % tt font for numpy

\topmargin -.5in
\textheight 9in
\oddsidemargin -.25in
\evensidemargin -.25in
\textwidth 7in

\begin{document}

% ========== Edit your name here
\author{Dr. Neil Eliot / Dr. Alun Moon}
\title{KF5004\\------\\Company Brief\\------\\The Implementation of a \texttt{DNS} infrastructure (\texttt{BIND9}) and Web Server Farm facilities
(\texttt{Apache, PHP, NFS, and MySQL})\\------}
\date{September 2019}
\maketitle

\newpage
\tableofcontents
\newpage

\section{Introduction}
A company requires a large-scale implementation of a \texttt{DNS} architecture with a \textbf{load balanced} web server farm and an \texttt{intranet} to fulfil its 
web infrastructure needs. The company also wishes to manage aspects of its computer assets and staff requirements through the \texttt{DNS} implementation. 
Below is a description of the company needs.

\section{Overview}

The company requires two \texttt{domain}s and one \texttt{subdomain} to fulfil its internal and external needs.

\begin{itemize}
    \item unn.co.uk
    \item tech.co.uk
    \item staff.unn.co.uk
\end{itemize}

\noindent The three zones fulfil different requirements based on how devices are addressed.\\

\noindent The company has a single \texttt{gateway} machine connecting the company to the internet and has 15 static \texttt{IP} addresses supplied 
by their \texttt{ISP}. 
These addresses allow the company to map machines directly to the internet from their internal network such as their company internet web server 
farm (details to follow). This external mapping is outside of the brief and is supplied for information only.\\\\

The company’s gateway \texttt{IP} Address is: \texttt{192.168.100.254}

\subsection{Domain - Purpose}

\subsubsection{\texttt{unn}}
This domain is managed by the technicians. The domain is used to direct traffic to the company’s externally facing web service and internally 
for the company’s internal network. The \texttt{zone} is also used for mapping of the main servers such as \texttt{NFS} and the central \texttt{MySQL} server. 

\subsubsection{\texttt{tech}}
This \texttt{zone} is managed by the technicians and is used by the technical staff so they can remotely access machines via a fixed asset number that 
the machines are given (The asset number remains with a machine while it is in use by the company). This \texttt{zone} should only be accessible from 
the technician’s machines that are in the range \texttt{192.168.170.0/24} and \texttt{192.168.180.0/24} and \texttt{192.168.190.0/24}

\subsubsection{\texttt{staff.unn}}
The \texttt{zone} is managed by the \textbf{personnel department} who have their own administrator who runs the \texttt{zone} in cooperation with the \texttt{IT} 
department. This \texttt{zone} is used to manage the movements of staff within the organisation and their locally shared resources. i.e. it is used to point 
to the machine they are using. Once a person is allocated a \texttt{FQDN} in the \texttt{staff.unn zone} it remains with them as long they work at the company. 
As staff move to new equipment (\texttt{PC}’s) the \texttt{IT} department inform personnel department who will change the \texttt{DNS} entry.

\section{Operational Implementation}
\subsection{Transaction Logging}
\subsection{Client \texttt{DNS} configuration}


\end{document}
