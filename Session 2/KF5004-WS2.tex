\documentclass[11pt]{article}
%\usepackage[framemethod=tikz]{mdframed}
\usepackage{tcolorbox}
\usepackage[os=win]{menukeys}
\newcommand{\numpy}{{\tt numpy}}    % tt font for numpy

\topmargin -.5in
\textheight 9in
\oddsidemargin -.25in
\evensidemargin -.25in
\textwidth 7in

\begin{document}

% ========== Edit your name here
\author{Dr. Neil Eliot / Dr. Alun Moon}
\title{KF5004\\------\\Workshop 1\\------\\Setting up a basic headless \texttt{Ubuntu Server}\\------}
\date{September 2019}
\maketitle

\medskip

\begin{center}
    \noindent\rule{8cm}{0.4pt}
\end{center}

% ========== Introduction

\noindent\textbf{Learning Outcomes:}
\begin{itemize}
    \item Understand what a Headless server is.
    \item Understand what a console interface is.
    \item Configure a server to support \texttt{ssh} and \texttt{telnetd}.
    \item Configure a server from a dynamic to a static \texttt{IP} Address
    \item Understand the location and purpose of the networking configuration files of a server.
\end{itemize}

\begin{center}
\noindent\rule{8cm}{0.4pt}
\end{center}

\noindent\textbf{The Exercise}\\

\noindent From the commands and process defined in the notes from the lecture carry out the following:\\

\noindent\textbf{NOTE:} Before you begin ensure you have the \texttt{PC} you are using connected to the Lab infrastructure and the machine can receive \texttt{IP} Addresses from the laboratory \texttt{DHCP} Server.

% ========== Begin answering questions here
\begin{enumerate}

\item Fill it in

\end{enumerate}

\end{document}