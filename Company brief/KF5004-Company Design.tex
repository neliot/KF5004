\documentclass[11pt]{article}
\usepackage{tcolorbox}
\usepackage{textcomp}
\usepackage[os=win]{menukeys}
\usepackage{pdflscape}
\usepackage{rotating}
\usepackage{listings}
\usepackage{titling}

\newcommand{\numpy}{{\tt numpy}}    % tt font for numpy

\topmargin -.5in
\textheight 9in
\oddsidemargin -.25in
\evensidemargin -.25in
\textwidth 7in

\pretitle{%
  \begin{center}
  \LARGE
  \includegraphics[width=15cm]{../images/logo.png}\\[\bigskipamount]
}
\posttitle{\end{center}}

\begin{document}

% ========== Edit your name here
\author{Dr. Neil Eliot / Dr. Alun Moon}
\title{KF5004\\------\\Network Design\\------\\The Implementation of a \texttt{DNS} infrastructure (\texttt{BIND9}) and Web Server Farm facilities
(\texttt{Apache, PHP, NFS, and MySQL})\\------}
\date{September 2019}
\maketitle

\newpage
\tableofcontents
\newpage

\section{\texttt{Revisions}}
\begin{table}[ht]
    \begin{tabular}{|p{1.5cm}|p{11cm}|p{2cm}|p{1.5cm}|} 
      \hline
      \multicolumn{1}{|c|}{Version} & \multicolumn{1}{|c|}{Description} & \multicolumn{1}{|c|}{Date}& \multicolumn{1}{|c|}{Author} \\ 
      \hline
      1.0 & Initial Schematic design. This is not a network diagram! It is indicative of the company deployment.& 03/06/2107 & NE\\
      \hline
      1.1 & Updated requirements for subdomain of staff.uun and updated configuration file structures and locations.& 01/09/2108 & NE\\
      \hline
      1.2 & Updated subdomain configuration to allow zone transfer to primary 1.& 10/12/2108 & NE\\
      \hline
    \end{tabular}
\end{table}
\newpage

\begin{landscape}
        \begin{figure}
        \begin{center}
            \includegraphics[width=1\linewidth]{Schematic.png}
            \caption{Schematic}
        \end{center}
      \end{figure}
\end{landscape}

\section{\texttt{Primary Server 1}}
\begin{table}[ht]
    \begin{tabular}{|p{17.7cm}|} 
        \hline
        \texttt{\textbf{/etc/netplan/10-cloud-init.yaml}}\\ 
        \hline
        \lstset{
                basicstyle=\scriptsize\ttfamily,
              }
              \begin{lstlisting}
# NETWORK CONFIG
network:
    version: 2
    ethernets:
      ens33:
        addresses: [192.168.101.28/16]
        gateway4: 192.168.100.254
        nameservers:
          search: [tech.co.uk, unn.co.uk]
          addresses: [8.8.8.8, 8.8.4.4]
        \end{lstlisting}\\
        \hline
    \end{tabular}
\end{table}

\begin{table}[ht]
    \begin{tabular}{|p{17.7cm}|} 
        \hline
        \texttt{\textbf{/etc/bind/named.conf.local}}\\ 
        \hline
        \lstset{
                basicstyle=\scriptsize\ttfamily,
              }
              \begin{lstlisting}
// 
// Do any local configuration here //
// Consider adding the 1918 zones here, if they are not used in your
// organization
// include "/etc/bind/zones.rfc1918";
                
zone "unn.co.uk" {
    type master;
    file "/etc/bind/db.unn.co.uk";
    allow-transfer {192.168.101.29; 192.168.101.30;};
    allow-query {192.168.101.29; 192.168.101.30;};
};
                
zone "tech.co.uk" {
    type master;
    file "/etc/bind/db.tech.co.uk";
    allow-transfer {192.168.101.29; 192.168.101.30;};
    allow-query {192.168.101.29; 192.168.101.30;};
};
                
zone "staff.unn.co.uk" {
    type slave;
    masters {192.168.101.27};
    allow-transfer {192.168.101.29; 192.168.101.30;};
    allow-query {192.168.101.29; 192.168.101.30;};
};
                
zone "168.192.in-addr.arpa" {
    type master;
    file "/etc/bind/db.168.192.in-addr.arpa";
    allow-transfer {192.168.101.29; 192.168.101.30;};
    allow-query {192.168.101.29; 192.168.101.30;};
};
        \end{lstlisting}\\
        \hline
    \end{tabular}
\end{table}

\begin{table}[ht]
    \begin{tabular}{|p{17.7cm}|} 
        \hline
        \texttt{\textbf{/etc/bind/named.conf.options}}\\ 
        \hline
        \lstset{
                basicstyle=\scriptsize\ttfamily,
              }
              \begin{lstlisting}
options {
    directory "/var/cache/bind";
            
    // If there is a firewall between you and nameservers you want
    // to talk to, you may need to fix the firewall to allow multiple
    // ports to talk.  See http://www.kb.cert.org/vuls/id/800113
            
    // If your ISP provided one or more IP addresses for stable 
    // nameservers, you probably want to use them as forwarders.  
    // Uncomment the following block, and insert the addresses replacing 
    // the all-0's placeholder.
            
    // forwarders {
    //         8.8.8.8;
    // };
            
    //========================================================================
    // If BIND logs error messages about the root key being expired,
    // you will need to update your keys.  See https://www.isc.org/bind-keys
    //========================================================================
    dnssec-validation auto;                                              
    auth-nxdomain no;    # conform to RFC1035
};
        \end{lstlisting}\\
        \hline
    \end{tabular}
\end{table}   

\begin{table}[ht]
    \begin{tabular}{|p{17.7cm}|} 
        \hline
        \texttt{\textbf{/etc/bind/db.unn.co.uk}}\\ 
        \hline
        \lstset{
                basicstyle=\scriptsize\ttfamily,
              }
              \begin{lstlisting}
$TTL    43200 
$ORIGIN unn.co.uk.
@  1d  IN  SOA ns1.unn.co.uk. hostmaster.unn.co.uk. (
                2017090807 ; sn
                172800     ; ref
                900        ; ret
                1209600    ; exp
                3600       ; nx
            )
@                           IN       NS                 ns1.unn.co.uk.
                            IN       NS                 ns2.unn.co.uk.
                            IN       MX      10         mail.microsoft.com.
                ns1         IN       A                  192.168.101.29
                ns2         IN       A                  192.168.101.30
                ns3         IN       A                  192.168.101.28
                web         IN       A                  192.168.101.40
                www         IN       A                  192.168.101.2
                            IN       A                  192.168.101.3
                            IN       A                  192.168.101.4
                ftp         IN       CNAME              web
                intranet    IN       CNAME              web
                mysql       IN       A                  192.168.101.5
                nfs1        IN       A                  192.168.101.6
                nfs2        IN       A                  192.168.101.7
        \end{lstlisting}\\
        \hline
    \end{tabular}
\end{table} 

\begin{table}[ht]
    \begin{tabular}{|p{17.7cm}|} 
        \hline
        \texttt{\textbf{/etc/bind/db.tech.co.uk}}\\ 
        \hline
        \lstset{
                basicstyle=\scriptsize\ttfamily,
              }
              \begin{lstlisting}
$TTL    0 
$ORIGIN tech.co.uk.
@  1d  IN  SOA ns1.unn.co.uk. hostmaster.unn.co.uk. (
                    2017092307 ; sn
                    1800       ; ref
                    900        ; ret
                    604800     ; exp
                    14400      ; nx
                )
@                IN        NS               ns1.unn.co.uk.
                 IN        NS               ns2.unn.co.uk.
A12345678        IN        A                192.168.101.40
web              IN        CNAME            A12345678
A12345679        IN        A                192.168.101.2
www1             IN        CNAME            A12345679
A12345680        IN        A                192.168.101.3
www2             IN        CNAME            A12345680
A12345681        IN        A                192.168.101.4
www3             IN        CNAME            A12345681
www              IN        A                192.168.101.2
                 IN        A                192.168.101.3
                 IN        A                192.168.101.4
A12345682        IN        A                192.168.101.5
mysql            IN        CNAME            A12345682
A12345683        IN        A                192.168.101.6
nfs1             IN        CNAME            A12345683
A12345684        IN        A                192.168.101.7
nfs2             IN        CNAME            A12345684
A12345685        IN        A                192.168.101.17
A12345686        IN        A                192.168.101.18
A12345687        IN        A                192.168.101.19
A12345688        IN        A                192.168.101.20
A12345689        IN        A                192.168.101.21
A12345690        IN        A                192.168.101.22
A12345691        IN        A                192.168.101.23
A12345692        IN        A                192.168.101.24
A12345693        IN        A                192.168.101.25
A12345694        IN        A                192.168.101.26
A12345695        IN        A                192.168.101.27
A12345696        IN        A                192.168.101.28
A12345697        IN        A                192.168.101.29
A12345698        IN        A                192.168.101.30
        \end{lstlisting}\\
        \hline
    \end{tabular}
\end{table}

\begin{table}[ht]
    \begin{tabular}{|p{17.7cm}|} 
        \hline
        \texttt{\textbf{/etc/bind/db.168.192.in-addr.arpa}}\\ 
        \hline
        \lstset{
                basicstyle=\scriptsize\ttfamily,
              }
              \begin{lstlisting}
$TTL 604800
$ORIGIN 168.192.in-addr.arpa.
@  1d        IN SOA        ns1.unn.co.uk. hostmaster.unn.co.uk. (
                            2015092303 ; sn
                            604800     ; ref
                            86400      ; ret
                            2419200    ; exp
                            604800     ; nx
                    )
                       NS     ns1.unn.co.uk.
                       NS     ns2.unn.co.uk.
2.101                  PTR    www.tech.co.uk.
3.101                  PTR    www.tech.co.uk.
4.101                  PTR    www.tech.co.uk.
5.101                  PTR    mysql.unn.co.uk.
        \end{lstlisting}\\
        \hline
    \end{tabular}
\end{table}

\clearpage
\section{\texttt{Primary Server 2}}
\begin{table}[ht]
    \begin{tabular}{|p{17.7cm}|} 
        \hline
        \texttt{\textbf{/etc/netplan/10-cloud-init.yaml}}\\ 
        \hline
        \lstset{
                basicstyle=\scriptsize\ttfamily,
              }
              \begin{lstlisting}
# NETWORK CONFIG
network:
    version: 2
    ethernets:
        ens33:
            addresses: [192.168.101.27/16]
            gateway4: 192.168.100.254
            nameservers:
                search: [tech.co.uk, unn.co.uk]
                addresses: [192.168.100.1,8.8.4.4]
        \end{lstlisting}\\
        \hline
    \end{tabular}
\end{table}

\begin{table}[ht]
    \begin{tabular}{|p{17.7cm}|} 
        \hline
        \texttt{\textbf{/etc/bind/named.conf.local}}\\ 
        \hline
        \lstset{
                basicstyle=\scriptsize\ttfamily,
              }
              \begin{lstlisting}
// 
// Do any local configuration here //
// Consider adding the 1918 zones here, if they are not used in your
// organization
// include "/etc/bind/zones.rfc1918";
                
zone "staff.unn.co.uk" {
        type master;
        file "/etc/bind/db.staff.unn.co.uk";
        allow-transfer {192.168.101.28;};
        allow-query{192.168.101.28;};
        notify no;
};
        \end{lstlisting}\\
        \hline
    \end{tabular}
\end{table}

\begin{table}[ht]
    \begin{tabular}{|p{17.7cm}|} 
        \hline
        \texttt{\textbf{/etc/bind/named.conf.options}}\\ 
        \hline
        \lstset{
                basicstyle=\scriptsize\ttfamily,
              }
              \begin{lstlisting}
options {
    directory "/var/cache/bind";
            
    // If there is a firewall between you and nameservers you want
    // to talk to, you may need to fix the firewall to allow multiple
    // ports to talk.  See http://www.kb.cert.org/vuls/id/800113
            
    // If your ISP provided one or more IP addresses for stable 
    // nameservers, you probably want to use them as forwarders.  
    // Uncomment the following block, and insert the addresses replacing 
    // the all-0's placeholder.
            
    // forwarders {
    //         8.8.8.8;
    // };
            
    //========================================================================
    // If BIND logs error messages about the root key being expired,
    // you will need to update your keys.  See https://www.isc.org/bind-keys
    //========================================================================
    dnssec-validation auto;                                              
    auth-nxdomain no;    # conform to RFC1035
};
        \end{lstlisting}\\
        \hline
    \end{tabular}
\end{table}

\begin{table}[ht]
    \begin{tabular}{|p{17.7cm}|} 
        \hline
        \texttt{\textbf{/etc/bind/db.staff.unn.co.uk}}\\ 
        \hline
        \lstset{
                basicstyle=\scriptsize\ttfamily,
              }
              \begin{lstlisting}
$TTL    43200 
$ORIGIN staff.unn.co.uk.
@  1d  IN  SOA ns1.unn.co.uk. hostmaster.unn.co.uk. (
                        2017090801 ; sn
                        172800     ; ref
                        900        ; ret
                        1209600    ; exp
                        3600       ; nx
                )
@                 IN        NS               ns1.unn.co.uk.
                  IN        NS               ns2.unn.co.uk.
ns4               IN        A                192.168.101.27
web               IN        A                192.168.101.40
fb                IN        A                192.168.101.17                                
ws                IN        A                192.168.101.18                                
dd                IN        A                192.168.101.19                                
ne                IN        A                192.168.101.20                        
ly                IN        A                192.168.101.21
        \end{lstlisting}\\
        \hline
    \end{tabular}
\end{table}

\clearpage
\section{\texttt{Secondary Server 1}}

\begin{table}[ht]
    \begin{tabular}{|p{17.7cm}|} 
        \hline
        \texttt{\textbf{/etc/netplan/10-cloud-init.yaml}}\\ 
        \hline
        \lstset{
                basicstyle=\scriptsize\ttfamily,
              }
              \begin{lstlisting}
# NETWORK CONFIG
network:
    version: 2
    ethernets:
        ens33:
            addresses: [192.168.101.29/16]
            gateway4: 192.168.100.254
            nameservers:
                search: [tech.co.uk, unn.co.uk]
                addresses: [8.8.8.8, 8.8.4.4]
        \end{lstlisting}\\
        \hline
    \end{tabular}
\end{table}

\begin{table}[ht]
    \begin{tabular}{|p{17.7cm}|} 
        \hline
        \texttt{\textbf{/etc/bind/named.conf.options}}\\ 
        \hline
        \lstset{
                basicstyle=\scriptsize\ttfamily,
              }
              \begin{lstlisting}
options {
    directory "/var/cache/bind";
            
    // If there is a firewall between you and nameservers you want
    // to talk to, you may need to fix the firewall to allow multiple
    // ports to talk.  See http://www.kb.cert.org/vuls/id/800113
            
    // If your ISP provided one or more IP addresses for stable 
    // nameservers, you probably want to use them as forwarders.  
    // Uncomment the following block, and insert the addresses replacing 
    // the all-0's placeholder.
            
    forwarders {
            192.168.101.254;
            8.8.8.8;
    };
            
    //========================================================================
    // If BIND logs error messages about the root key being expired,
    // you will need to update your keys.  See https://www.isc.org/bind-keys
    //========================================================================
    dnssec-validation auto;                                              
    auth-nxdomain no;    # conform to RFC1035
    };
        \end{lstlisting}\\
        \hline
    \end{tabular}
\end{table}

\begin{table}[ht]
    \begin{tabular}{|p{17.7cm}|} 
        \hline
        \texttt{\textbf{/etc/bind/named.conf.local}}\\ 
        \hline
        \lstset{
                basicstyle=\scriptsize\ttfamily,
              }
              \begin{lstlisting}
// 
// Do any local configuration here //
// Consider adding the 1918 zones here, if they are not used in your
// organization
// include "/etc/bind/zones.rfc1918";
                
zone "unn.co.uk" {
        type slave;
        file "db.unn.co.uk";
        masters {192.168.101.28;};
        allow-query {localnets;};
};
                
zone "tech.co.uk" {
        type slave;
        file "db.tech.co.uk";
        masters {192.168.101.28;};
        allow-query {localnets;};
};
                
zone "staff.unn.co.uk" {
        type slave;
        masters {192.168.101.28;};
        allow-query {localnets;};
};
                
zone "168.192.in-addr.arpa" {
        type slave;
        file "db.168.192.in-addr.arpa";
        masters {192.168.101.28;};
        allow-query {localnets;};
};
                
        \end{lstlisting}\\
        \hline
    \end{tabular}
\end{table}

\clearpage

\section{\texttt{Secondary Server 2}}
\begin{table}[ht]
    \begin{tabular}{|p{17.7cm}|} 
        \hline
        \texttt{\textbf{/etc/netplan/10-cloud-init.yaml}}\\ 
        \hline
        \lstset{
                basicstyle=\scriptsize\ttfamily,
              }
              \begin{lstlisting}
# NETWORK CONFIG
network:
    version: 2
    ethernets:
        ens33:
            addresses: [192.168.101.30/16]
            gateway4: 192.168.100.254
            nameservers:
                search: [tech.co.uk, unn.co.uk]
                addresses: [192.168.100.254, 8.8.4.4]
                
        \end{lstlisting}\\
        \hline
    \end{tabular}
\end{table}

\begin{table}[ht]
    \begin{tabular}{|p{17.7cm}|} 
        \hline
        \texttt{\textbf{/etc/bind/named.conf.options}}\\ 
        \hline
        \lstset{
                basicstyle=\scriptsize\ttfamily,
              }
              \begin{lstlisting}
options {
    directory "/var/cache/bind";
            
    // If there is a firewall between you and nameservers you want
    // to talk to, you may need to fix the firewall to allow multiple
    // ports to talk.  See http://www.kb.cert.org/vuls/id/800113
            
    // If your ISP provided one or more IP addresses for stable 
    // nameservers, you probably want to use them as forwarders.  
    // Uncomment the following block, and insert the addresses replacing 
    // the all-0's placeholder.
            
    forwarders {
        192.168.101.254;
        8.8.8.8;
    };
            
    //========================================================================
    // If BIND logs error messages about the root key being expired,
    // you will need to update your keys.  See https://www.isc.org/bind-keys
    //========================================================================
    dnssec-validation auto;                                              
    auth-nxdomain no;    # conform to RFC1035
};
        \end{lstlisting}\\
        \hline
    \end{tabular}
\end{table}

\begin{table}[ht]
    \begin{tabular}{|p{17.7cm}|} 
        \hline
        \texttt{\textbf{/etc/bind/named.conf.local}}\\ 
        \hline
        \lstset{
                basicstyle=\scriptsize\ttfamily,
              }
              \begin{lstlisting}
// 
// Do any local configuration here //
// Consider adding the 1918 zones here, if they are not used in your
// organization
// include "/etc/bind/zones.rfc1918";
                
zone "unn.co.uk" {
    type slave;
    file "db.unn.co.uk";
    masters {192.168.101.28;};
    allow-query {localnets;};
};
                
zone "tech.co.uk" {
    type slave;
    file "db.tech.co.uk";
    masters {192.168.101.28;};
    allow-query {localnets;};
};
                
zone "staff.unn.co.uk" {
    type slave;
    masters {192.168.101.28;};
    allow-query {localnets;};
};
                
zone "168.192.in-addr.arpa" {
    type slave;
    file "db.168.192.in-addr.arpa";
    masters {192.168.101.28;};
    allow-query {localnets;};
};

        \end{lstlisting}\\
        \hline
    \end{tabular}
\end{table}

\clearpage

\section{\texttt{Apache Server 1}}
\begin{table}[ht]
    \begin{tabular}{|p{17.7cm}|} 
        \hline
        \texttt{\textbf{/etc/netplan/10-cloud-init.yaml}}\\ 
        \hline
        \lstset{
                basicstyle=\scriptsize\ttfamily,
              }
              \begin{lstlisting}
# NETWORK CONFIG
network:
    version: 2
    ethernets:
        ens33:
            addresses: [192.168.101.2/16]
            gateway4: 192.168.100.254
            nameservers:
                search: [tech.co.uk, unn.co.uk]
                addresses: [192.168.101.29, 192.168.101.30]                
        \end{lstlisting}\\
        \hline
    \end{tabular}
\end{table}

\begin{table}[ht]
    \begin{tabular}{|p{17.7cm}|} 
        \hline
        \texttt{\textbf{/etc/fstab}}\\ 
        \hline
        \lstset{
                basicstyle=\scriptsize\ttfamily,
              }
              \begin{lstlisting}
# /etc/fstab: static file system information.
#
# Use 'blkid' to print the universally unique identifier for a
# device; this may be used with UUID= as a more robust way to name devices
# that works even if disks are added and removed. See fstab(5).
#
 <file system> <mount point>   <type>  <options>       <dump>  <pass>
# / was on /dev/sda1 during installation
UUID=fdae0e93-c8ec-44aa-a462-5b62af1c4277 /               ext4    errors=remount-ro 0       1
# swap was on /dev/sda5 during installation
UUID=3b24a428-f71a-4964-92d0-d812c0b6e107 none            swap    sw                0       0
/dev/fd0                       /media/floppy0        auto rw,user,noauto,exec,utf8  0       0
#MULTIPLE ENTRIES SELECT APPROPRIATELY
#nfs1.tech.co.uk:/var/content /var/www/html/wp-content nfs  rw,soft,timeo=101,intr              0 0
#:/var/content /var/www/html/wp-content
#nfs1.tech.co.uk:/var/content /var/www/html nfs  rw,soft,timeo=101,intr                      0 0
#nfs1.tech.co.uk /var/content /var/www/html/wp-content nfs  rw,soft,timeo=101,intr              0 0
#nfs1.tech.co.uk:/var/content 192.168.101.4:/var/www/html/wp-content nfs rw,soft,timeo=101,intr 0 0           
        \end{lstlisting}\\
        \hline
    \end{tabular}
\end{table}

\clearpage

\section{\texttt{Apache Server 2}}
\begin{table}[ht]
    \begin{tabular}{|p{17.7cm}|} 
        \hline
        \texttt{\textbf{/etc/netplan/10-cloud-init.yaml}}\\ 
        \hline
        \lstset{
                basicstyle=\scriptsize\ttfamily,
              }
              \begin{lstlisting}
# NETWORK CONFIG
network:
    version: 2
    ethernets:
        ens33:
            addresses: [192.168.101.3/16]
            gateway4: 192.168.100.254
            nameservers:
                search: [tech.co.uk, unn.co.uk]
                addresses: [192.168.101.29, 192.168.101.30]                
        \end{lstlisting}\\
        \hline
    \end{tabular}
\end{table}

\begin{table}[ht]
    \begin{tabular}{|p{17.7cm}|} 
        \hline
        \texttt{\textbf{/etc/fstab}}\\ 
        \hline
        \lstset{
                basicstyle=\scriptsize\ttfamily,
              }
              \begin{lstlisting}
# /etc/fstab: static file system information.
#
# Use 'blkid' to print the universally unique identifier for a
# device; this may be used with UUID= as a more robust way to name devices
# that works even if disks are added and removed. See fstab(5).
#
 <file system> <mount point>   <type>  <options>       <dump>  <pass>
# / was on /dev/sda1 during installation
UUID=fdae0e93-c8ec-44aa-a462-5b62af1c4245 /               ext4    errors=remount-ro 0       1
# swap was on /dev/sda5 during installation
UUID=3b24a428-f71a-4964-92d0-d812c0b6e109 none            swap    sw                0       0
/dev/fd0                       /media/floppy0        auto rw,user,noauto,exec,utf8  0       0
#MULTIPLE ENTRIES SELECT APPROPRIATELY
#nfs1.tech.co.uk:/var/content /var/www/html/wp-content nfs  rw,soft,timeo=101,intr              0 0
#:/var/content /var/www/html/wp-content
#nfs1.tech.co.uk:/var/content /var/www/html nfs  rw,soft,timeo=101,intr                      0 0
#nfs1.tech.co.uk /var/content /var/www/html/wp-content nfs  rw,soft,timeo=101,intr              0 0
#nfs1.tech.co.uk:/var/content 192.168.101.4:/var/www/html/wp-content nfs rw,soft,timeo=101,intr 0 0           
        \end{lstlisting}\\
        \hline
    \end{tabular}
\end{table}

\clearpage

\section{\texttt{Apache Server 3}}
\begin{table}[ht]
    \begin{tabular}{|p{17.7cm}|} 
        \hline
        \texttt{\textbf{/etc/netplan/10-cloud-init.yaml}}\\ 
        \hline
        \lstset{
                basicstyle=\scriptsize\ttfamily,
              }
              \begin{lstlisting}
# NETWORK CONFIG
network:
    version: 2
    ethernets:
        ens33:
            addresses: [192.168.101.4/16]
            gateway4: 192.168.100.254
            nameservers:
                search: [tech.co.uk, unn.co.uk]
                addresses: [192.168.101.29, 192.168.101.30]                
        \end{lstlisting}\\
        \hline
    \end{tabular}
\end{table}

\begin{table}[ht]
    \begin{tabular}{|p{17.7cm}|} 
        \hline
        \texttt{\textbf{/etc/fstab}}\\ 
        \hline
        \lstset{
                basicstyle=\scriptsize\ttfamily,
              }
              \begin{lstlisting}
# /etc/fstab: static file system information.
#
# Use 'blkid' to print the universally unique identifier for a
# device; this may be used with UUID= as a more robust way to name devices
# that works even if disks are added and removed. See fstab(5).
#
 <file system> <mount point>   <type>  <options>       <dump>  <pass>
# / was on /dev/sda1 during installation
UUID=fdae0e93-c8ec-44aa-a462-5b62af1c4259 /               ext4    errors=remount-ro 0       1
# swap was on /dev/sda5 during installation
UUID=3b24a428-f71a-4964-92d0-d812c0b6e110 none            swap    sw                0       0
/dev/fd0                       /media/floppy0        auto rw,user,noauto,exec,utf8  0       0
#MULTIPLE ENTRIES SELECT APPROPRIATELY
#nfs1.tech.co.uk:/var/content /var/www/html/wp-content nfs  rw,soft,timeo=101,intr              0 0
#:/var/content /var/www/html/wp-content
#nfs1.tech.co.uk:/var/content /var/www/html nfs  rw,soft,timeo=101,intr                      0 0
#nfs1.tech.co.uk /var/content /var/www/html/wp-content nfs  rw,soft,timeo=101,intr              0 0
#nfs1.tech.co.uk:/var/content 192.168.101.4:/var/www/html/wp-content nfs rw,soft,timeo=101,intr 0 0           
        \end{lstlisting}\\
        \hline
    \end{tabular}
\end{table}

\clearpage

\section{\texttt{MySQL Server}}
\begin{table}[ht]
    \begin{tabular}{|p{17.7cm}|} 
        \hline
        \texttt{\textbf{/etc/netplan/10-cloud-init.yaml}}\\ 
        \hline
        \lstset{
                basicstyle=\scriptsize\ttfamily,
              }
              \begin{lstlisting}
# NETWORK CONFIG
network:
    version: 2
    ethernets:
        ens33:
            addresses: [192.168.101.5/16]
            gateway4: 192.168.100.254
            nameservers:
                search: [tech.co.uk, unn.co.uk]
                addresses: [192.168.101.29, 192.168.101.30]                
        \end{lstlisting}\\
        \hline
    \end{tabular}
\end{table}

\begin{table}[ht]
    \begin{tabular}{|p{17.7cm}|} 
        \hline
        \texttt{\textbf{/etc/mysql/mysql.conf.d/mysqld.cnf} - \texttt{(1)}}\\ 
        \hline
        \lstset{
                basicstyle=\scriptsize\ttfamily,
              }
              \begin{lstlisting}
# The following values assume you have at least 32M ram

[mysqld_safe]
socket          = /var/run/mysqld/mysqld.sock
nice            = 0

[mysqld]
#
# * Basic Settings
#
user            = mysql
pid-file        = /var/run/mysqld/mysqld.pid
socket          = /var/run/mysqld/mysqld.sock
port            = 3306
basedir         = /usr
datadir         = /var/lib/mysql
tmpdir          = /tmp
lc-messages-dir = /usr/share/mysql
skip-external-locking
#
# Instead of skip-networking the default is now to listen only on
# localhost which is more compatible and is not less secure.
#MULTIPLE ENTRIES SELECT APPROPRIATELY
#bind-address=::
#bind-address=0.0.0.0
#bind-address=127.0.0.1
#bind-address=www.unn.ac.uk
#bind-address=192.168.101.1
#bind-address=192.168.101.2 192.168.101.3 192.168.101.4
#
# * Fine Tuning
#
key_buffer_size         = 16M
max_allowed_packet      = 16M
thread_stack            = 192K
thread_cache_size       = 8
# This replaces the startup script and checks MyISAM tables if needed
# the first time they are touched
myisam-recover-options  = BACKUP
#max_connections        = 101
#table_cache            = 64
#thread_concurrency     = 10
#
# * Query Cache Configuration
#
# This replaces the startup script and checks MyISAM tables if needed
# the first time they are touched
myisam-recover-options  = BACKUP
#max_connections        = 101
#table_cache            = 64
#thread_concurrency     = 10
#
# * Query Cache Configuration
#
query_cache_limit       = 1M
query_cache_size        = 16M
#
# * Logging and Replication
#
# Both location gets rotated by the cronjob.
# Be aware that this log type is a performance killer.
# As of 5.1 you can enable the log at runtime!
#general_log_file        = /var/log/mysql/mysql.log
#general_log             = 1
#                               
        \end{lstlisting}\\
        \hline
    \end{tabular}
\end{table}

\begin{table}[ht]
    \begin{tabular}{|p{17.7cm}|} 
        \hline
        \texttt{\textbf{/etc/mysql/mysql.conf.d/mysqld.cnf} - \texttt{(2)}}\\ 
        \hline
        \lstset{
                basicstyle=\scriptsize\ttfamily,
              }
              \begin{lstlisting}
#
# Error log - should be very few entries.
#
log_error = /var/log/mysql/error.log
#
# Here you can see queries with especially long duration
#log_slow_queries       = /var/log/mysql/mysql-slow.log
#long_query_time = 2
#log-queries-not-using-indexes
#
# The following can be used as easy to replay backup logs or for replication.
# note: if you are setting up a replication slave, see README.Debian about
#       other settings you may need to change.
#server-id              = 1
#log_bin                        = /var/log/mysql/mysql-bin.log
expire_logs_days        = 10
max_binlog_size   = 101M
#binlog_do_db           = include_database_name
#binlog_ignore_db       = include_database_name
#
# * InnoDB
#
# InnoDB is enabled by default with a 10MB datafile in /var/lib/mysql/.
# Read the manual for more InnoDB related options. There are many!
#
# * Security Features
#
# Read the manual, too, if you want chroot!
# chroot = /var/lib/mysql/
#
#
# InnoDB is enabled by default with a 10MB datafile in /var/lib/mysql/.
# Read the manual for more InnoDB related options. There are many!
#
# * Security Features
#
# Read the manual, too, if you want chroot!
# chroot = /var/lib/mysql/
#
# For generating SSL certificates I recommend the OpenSSL GUI "tinyca".
#
# ssl-ca=/etc/mysql/cacert.pem
# ssl-cert=/etc/mysql/server-cert.pem
# ssl-key=/etc/mysql/server-key.pem                               
        \end{lstlisting}\\
        \hline
    \end{tabular}
\end{table}

\clearpage

\section{\texttt{Intranet Server}}
\begin{table}[ht]
    \begin{tabular}{|p{17.7cm}|} 
        \hline
        \texttt{\textbf{/etc/netplan/10-cloud-init.yaml}}\\ 
        \hline
        \lstset{
                basicstyle=\scriptsize\ttfamily,
              }
              \begin{lstlisting}
# NETWORK CONFIG
network:
    version: 2
    ethernets:
        ens33:
            addresses: [192.168.101.40/16]
            gateway4: 192.168.100.254
            nameservers:
                search: [tech.co.uk, unn.co.uk]
                addresses: [192.168.101.29, 192.168.101.30]                
        \end{lstlisting}\\
        \hline
    \end{tabular}
\end{table}

\begin{table}[ht]
    \begin{tabular}{|p{17.7cm}|} 
        \hline
        \texttt{\textbf{/etc/apache2/mods-available/php7.2.conf}}\\ 
        \hline
        \lstset{
                basicstyle=\scriptsize\ttfamily,
              }
              \begin{lstlisting}
<FilesMatch ".+\.ph(p[345]?|t|tml)$">
    SetHandler application/x-httpd-php
</FilesMatch>
<FilesMatch ".+\.phps$">
    SetHandler application/x-httpd-php-source
    # Deny access to raw php sources by default
    # To re-enable it's recommended to enable access to the files
    # only in specific virtual host or directory
    Order Deny,Allow
    Deny from all
</FilesMatch>
# Deny access to files without filename (e.g. '.php')
<FilesMatch "^\.ph(p[345]?|t|tml|ps)$">
    Order Deny,Allow
    Deny from all
</FilesMatch>

# Running PHP scripts in user directories is disabled by default
# 
# To re-enable PHP in user directories comment the following lines
# (from <IfModule ...> to </IfModule>.) Do NOT set it to On as it
# prevents .htaccess files from disabling it.
<IfModule mod_userdir.c>
    <Directory /home/*/public_html>
        php_admin_flag engine Off
    </Directory>
    <Directory /home/ceo/public_html>
        php_admin_flag engine On
    </Directory>
</IfModule>                
        \end{lstlisting}\\
        \hline
    \end{tabular}
\end{table}

\begin{table}[ht]
    \begin{tabular}{|p{17.7cm}|} 
        \hline
    \texttt{\textbf{/etc/mysql/mysql.conf.d/mysqld.cnf} - (\texttt{1})}\\ 
        \hline
        \lstset{
                basicstyle=\scriptsize\ttfamily,
        }
            \begin{lstlisting}
# The following values assume you have at least 32M ram

[mysqld_safe]
socket          = /var/run/mysqld/mysqld.sock
nice            = 0

[mysqld]
#
# * Basic Settings
#
user            = mysql
pid-file        = /var/run/mysqld/mysqld.pid
socket          = /var/run/mysqld/mysqld.sock
port            = 3306
basedir         = /usr
datadir         = /var/lib/mysql
tmpdir          = /tmp
lc-messages-dir = /usr/share/mysql
skip-external-locking
#
# Instead of skip-networking the default is now to listen only on
# localhost which is more compatible and is not less secure.
#MULTIPLE ENTRIES SELECT APPROPRIATELY
#bind-address=::
#bind-address=0.0.0.0
#bind-address=127.0.0.1
#bind-address=www.unn.co.uk
#bind-address=192.168.101.1
#bind-address=192.168.101.2 192.168.101.3 192.168.101.4
#
# * Fine Tuning
#
key_buffer_size         = 16M
max_allowed_packet      = 16M
thread_stack            = 192K
thread_cache_size       = 8
# This replaces the startup script and checks MyISAM tables if needed
# the first time they are touched
myisam-recover-options  = BACKUP
#max_connections        = 101
#table_cache            = 64
#thread_concurrency     = 10
#
# * Query Cache Configuration
#
# This replaces the startup script and checks MyISAM tables if needed
# the first time they are touched
myisam-recover-options  = BACKUP
#max_connections        = 101
#table_cache            = 64
#thread_concurrency     = 10
#
# * Query Cache Configuration
#
query_cache_limit       = 1M
query_cache_size        = 16M
            \end{lstlisting}\\
        \hline
    \end{tabular}
\end{table}

\begin{table}[ht]
    \begin{tabular}{|p{17.7cm}|} 
        \hline
    \texttt{\textbf{/etc/mysql/mysql.conf.d/mysqld.cnf} - (\texttt{2})}\\ 
        \hline
        \lstset{
                basicstyle=\scriptsize\ttfamily,
        }
            \begin{lstlisting}
#
# * Logging and Replication
#
# Both location gets rotated by the cronjob.
# Be aware that this log type is a performance killer.
# As of 5.1 you can enable the log at runtime!
#general_log_file        = /var/log/mysql/mysql.log
#general_log             = 1
query_cache_size        = 16M               
#
# Error log - should be very few entries.
#
log_error = /var/log/mysql/error.log
#
# Here you can see queries with especially long duration
#log_slow_queries       = /var/log/mysql/mysql-slow.log
#long_query_time = 2
#log-queries-not-using-indexes
#
# Error log - should be very few entries.
#
log_error = /var/log/mysql/error.log
#
# Here you can see queries with especially long duration
#log_slow_queries       = /var/log/mysql/mysql-slow.log
#long_query_time = 2
#log-queries-not-using-indexes
#
# The following can be used as easy to replay backup logs or for replication.
# note: if you are setting up a replication slave, see README.Debian about
#       other settings you may need to change.
#server-id              = 1
#log_bin                        = /var/log/mysql/mysql-bin.log
expire_logs_days        = 10
max_binlog_size   = 101M
#binlog_do_db           = include_database_name
#binlog_ignore_db       = include_database_name
#
# * InnoDB
#
# InnoDB is enabled by default with a 10MB datafile in /var/lib/mysql/.
# Read the manual for more InnoDB related options. There are many!
#
# * Security Features
#
# Read the manual, too, if you want chroot!
# chroot = /var/lib/mysql/
#
#
# InnoDB is enabled by default with a 10MB datafile in /var/lib/mysql/.
# Read the manual for more InnoDB related options. There are many!
#
# * Security Features
#
# Read the manual, too, if you want chroot!
# chroot = /var/lib/mysql/
#
# For generating SSL certificates I recommend the OpenSSL GUI "tinyca".
#
# ssl-ca=/etc/mysql/cacert.pem
# ssl-cert=/etc/mysql/server-cert.pem
# ssl-key=/etc/mysql/server-key.pem               
            \end{lstlisting}\\
        \hline
    \end{tabular}
\end{table}

\clearpage

\section{\texttt{NFS Server 1}}
\begin{table}[ht]
    \begin{tabular}{|p{17.7cm}|} 
        \hline
    \texttt{\textbf{/etc/netplan/10-cloud-init.yaml}}\\ 
        \hline
        \lstset{
                basicstyle=\scriptsize\ttfamily,
        }
            \begin{lstlisting}
# NETWORK CONFIG
network:
    version: 2
    ethernets:
      ens33:
        addresses: [192.168.101.6/16]
        gateway4: 192.168.100.254
        nameservers:
          search: [tech.co.uk, unn.co.uk]
          addresses: [192.168.101.29, 192.168.101.30]
            \end{lstlisting}\\
        \hline
    \end{tabular}
\end{table}

\begin{table}[ht]
    \begin{tabular}{|p{17.7cm}|} 
        \hline
    \texttt{\textbf{/etc/exports (nfs-kernel-server installed)}}\\ 
        \hline
        \lstset{
                basicstyle=\scriptsize\ttfamily,
        }
            \begin{lstlisting}
# /etc/exports: the access control list for filesystems which may be exported
#                to NFS clients.  See exports(5).
#
# Example for NFSv2 and NFSv3:
# /srv/homes       hostname1(rw,sync,no_subtree_check) hostname2(ro,sync,no_subtree_check)
#
# Example for NFSv4:
# /srv/nfs4        gss/krb5i(rw,sync,fsid=0,crossmnt,no_subtree_check)
# /srv/nfs4/homes  gss/krb5i(rw,sync,no_subtree_check)
#
#MULTIPLE ENTRIES SELECT APPROPRIATELY
#/var/content *(rw,sync,no_subtree_check)
#/var/content www?(rw,sync,no_subtree_check)
#/var/content www.unn.co.uk(rw,sync,no_subtree_check)
#/var/content www.tech.co.uk(rw,sync,no_subtree_check)
#/var/content 192.168.101.2(rw,sync,no_subtree_check)
            \end{lstlisting}\\
        \hline
    \end{tabular}
\end{table}

\clearpage

\section{\texttt{NFS Server 2}}
\begin{table}[ht]
    \begin{tabular}{|p{17.7cm}|} 
        \hline
    \texttt{\textbf{/etc/netplan/10-cloud-init.yaml}}\\ 
        \hline
        \lstset{
                basicstyle=\scriptsize\ttfamily,
        }
            \begin{lstlisting}
# NETWORK CONFIG
network:
    version: 2
    ethernets:
      ens33:
        addresses: [192.168.101.7/16]
        gateway4: 192.168.100.254
        nameservers:
          search: [tech.co.uk, unn.co.uk]
          addresses: [192.168.101.29, 192.168.101.30]
            \end{lstlisting}\\
        \hline
    \end{tabular}
\end{table}

\noindent \texttt{END}
\end{document}
