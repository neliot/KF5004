\documentclass[11pt]{article}
%\usepackage[framemethod=tikz]{mdframed}
\usepackage{tcolorbox}
\usepackage[os=win]{menukeys}
\newcommand{\numpy}{{\tt numpy}}    % tt font for numpy

\topmargin -.5in
\textheight 9in
\oddsidemargin -.25in
\evensidemargin -.25in
\textwidth 7in

\begin{document}

% ========== Edit your name here
\author{Dr. Neil Eliot / Dr. Alun Moon}
\title{KF5004\\------\\Workshop 2\\------\\Setting up a basic headless \texttt{Ubuntu Server}\\------}
\date{September 2019}
\maketitle

\medskip

\begin{center}
    \noindent\rule{8cm}{0.4pt}
\end{center}

% ========== Introduction

\noindent\textbf{Learning Outcomes:}
\begin{itemize}
    \item Understand what a Headless server is.
    \item Understand what a console interface is.
    \item Configure a server to support \texttt{ssh} and \texttt{telnetd}.
    \item Configure a server from a dynamic to a static \texttt{IP} Address
    \item Understand the location and purpose of the networking configuration files of a server.
\end{itemize}

\begin{center}
\noindent\rule{8cm}{0.4pt}
\end{center}

\begin{tcolorbox}[title={\textbf{Note:}}]
    Before you begin ensure you have the \texttt{PC} you are using connected to the Lab infrastructure and the machine can receive \texttt{IP} Addresses from the laboratory \texttt{DHCP} Server.
\end{tcolorbox}

\noindent\textbf{The Exercise}\\

\noindent \textbf{From the commands and process defined in the notes from the lecture carry out the following:\\}

% ========== Begin answering questions here
\begin{enumerate}
    \item Create a new Virtual Machine and carry out a basic install.
    \begin{itemize}
        \item Take the default settings in most cases.
        \begin{itemize}
            \item Take the default settings in most cases.
            \begin{itemize}
                \item Set your machine name to your \texttt{User ID}.
                \item Set Username to \texttt{student}
                \item Set Password to \texttt{northumbria}
                \item Set memory to \texttt{2Gb}
                \item Set Network card to Bridged.
            \end{itemize}
        \end{itemize} 
    \end{itemize}
    \item Identify the address the server has been given by the \texttt{DHCP} Server. 
        \begin{tcolorbox}[title={\textbf{Note:}}]
            At this stage you should close the Virtual Machine down and copy the folder as this will be your base machine for future work! To `shutdown' your server use the command: \texttt{sudo shutdown -h now} 
        \end{tcolorbox}
    \item Restart your machine to continue.
    \item Modify the server so that it has (from your list of provided settings) a static configurations of:
        \begin{itemize}
            \item \texttt{IP - <choose one from your batch>}
            \item \texttt{Subnet - 255.255.0.0 (/16)}
            \item \texttt{Gateway - 192.168.100.254}
            \item \texttt{DNS - 192.168.101.29}
                \begin{tcolorbox}[title={\textbf{Note:}}]
                    At this stage you are using the Virtual Lab name servers. Later you will be creating your own \texttt{DNS} architecture.
                \end{tcolorbox}
        \end{itemize}
    \item Add \texttt{ssh} services to your machine for remote access (\texttt{openssh-server}).
        \begin{tcolorbox}[title={\textbf{Note:}}]
            In the next section you should connect from a different machine on the network. This forces the traffic outside of the \texttt{TCP} stack and onto the network so \texttt{Wireshark} can capture the data.
        \end{tcolorbox}
    \end{enumerate}
    \noindent \textbf{Using the server you have created carry out the following:}
    \begin{enumerate}
        \setcounter{enumi}{5}
        \item Create a new Virtual Machine and carry out a basic install.
            \begin{itemize}
                \item Remotely connect to the server.
                    \begin{tcolorbox}[title={\textbf{Note:}}]
                        \texttt{putty} should be on all the lab \texttt{PC}'s Windows installation. (on the first connection you will need to accept the certificate key)\\~\\
                        If you are using \texttt{Ubuntu desktop} you should be able to connect to the server by using the \texttt{ssh} command from a \texttt{Linux} console (on the first connection you will need to accept the certificate key)
                    \end{tcolorbox}
            \end{itemize}
        \item Using \texttt{Wireshark} monitor the network and identify the communications to/from the virtual machine and identify the \texttt{IP} address and \texttt{MAC} address in the \texttt{IP} headers.
        \item Restart the server.
        \item Install \texttt{telnet} (\texttt{telnetd})
        \item Run \texttt{wireshark} and capture the traffic while you are logging in using either the \texttt{telnet} command in \texttt{Linux} or \texttt{telnet} protocol in \texttt{putty} (filter \texttt{tcp.port == 23}).
        \item Examine the traffic and identify the \texttt{TCP} transactions for the login.
            \begin{itemize}
                \item You will find that each letter of the username is show as a transmit from the client and then a transmit from the server.
                \item The password is not echoed you will see the characters only once.
            \end{itemize}
        \item Shutdown the server.
            \begin{itemize}
                \item \texttt{\$shutdown -h now}
            \end{itemize}
    \end{enumerate}
    \begin{tcolorbox}[title={\textbf{TIME PERMITTING:}}]
        Research how to change the password on your server and how to add another user.
    \end{tcolorbox}
\end{document}