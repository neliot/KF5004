\documentclass[11pt]{article}
%\usepackage[framemethod=tikz]{mdframed}
\usepackage{tcolorbox}
%\usepackage{figure}
\usepackage[os=win]{menukeys}
\newcommand{\numpy}{{\tt numpy}}    % tt font for numpy

\topmargin -.5in
\textheight 9in
\oddsidemargin -.25in
\evensidemargin -.25in
\textwidth 7in

\begin{document}

% ========== Edit your name here
\author{Dr. Neil Eliot / Dr. Alun Moon}
\title{KF5004\\------\\Workshop 9\\------\\Setting up \texttt{NFS}\\------}
\date{September 2019}
\maketitle

\newpage

% \begin{center}
%     \noindent\rule{8cm}{0.4pt}
% \end{center}

% ========== Introduction

\noindent\textbf{Learning Outcomes:}
\begin{itemize}
    \item Configure \texttt{NFS} server.
    \item Configure \texttt{NFS} client.
    \item Create shared content for websites.
\end{itemize}

% \begin{center}
% \noindent\rule{8cm}{0.4pt}
% \end{center}

\begin{tcolorbox}[title={\textbf{Important:}}]
    Before you begin, ensure the lab. \texttt{PC} you are using is connected to the off campus lab infrastructure and the machine receives an \texttt{IP} address from the lab. \texttt{DHCP} Server.
\end{tcolorbox}
\newpage

\noindent\textbf{The Exercise}\\
\begin{tcolorbox}[colback=blue!20]
    \noindent\textbf{From the commands and process defined in the notes from the lecture carry out the following:}
\end{tcolorbox}

% ========== Begin answering questions here

\begin{tcolorbox}[title={\textbf{NOTE:}}]
    In this workshop you will need your \texttt{DNS} and Web server architecture from the last practical.
\end{tcolorbox}

\begin{enumerate}
    \item Create a new \texttt{Virtual Server}.
        \begin{itemize}
            \item Install \texttt{NFS} server.
            \item Create a folder \texttt{/etc/content}
            \item Export the \texttt{/etc/content} folder to allow anyone to have read write access.
        \end{itemize}
    \item On the all the web servers:
        \begin{itemize}
            \item Install \texttt{NFS} client software.
            \item Create a mount point of \texttt{/var/www/html/content}
        \end{itemize}
    \item Mount the exported folder to all the webs servers using the \texttt{mount} command.
    \item On the \texttt{NFS} server add a text file to \texttt{/etc/content} called \texttt{`hello.txt'} and add your student id for content.
    \item \label{repeat1} List the contents of each web servers \texttt{/var/www/html/content} folder.
    \item \label{repeat2} From a web browser enter the \texttt{URL} - \texttt{http://www.student.co.uk/content/hello.txt}
    \item Unmount the content folder for all of the servers and repeat steps \ref{repeat1} and \ref{repeat2}.
    \item Add an entry to the \texttt{fstab} of each of the web servers so the \texttt{content} folder is automatically mounted.
    \item On each web server:
        \begin{itemize}
            \item Mount the \texttt{content} folder using the \texttt{mount} command and check the \texttt{NFS} share has been mounted correctly.
            \item Unmount the \texttt{content} folder and check the mounts have been removed.
            \item Reboot the server.
            \item Check the \texttt{content} folder has auto-mounted.
        \end{itemize}
    \item From a web browser navigate to the following sites:
        \begin{itemize}
            \item \texttt{http://www.student.co.uk/content/hello.txt}
            \item \texttt{http://www1.student.co.uk/content/hello.txt}
            \item \texttt{http://www2.student.co.uk/content/hello.txt}
            \item \texttt{http://www3.student.co.uk/content/hello.txt}
        \end{itemize}
\end{enumerate}    
END
\end{document}